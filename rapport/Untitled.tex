%%%%%%%%%%%%%%%%%%%%%%%%%%%%%%%%%%%%%%%%%
%
% LaTeX Template
% Version 1.0 (26/01/16)
%
%%%%%%%%%%%%%%%%%%%%%%%%%%%%%%%%%%%%%%%%%

\documentclass[11pt, letterpaper]{article}
\usepackage[latin1]{inputenc}
\usepackage[T1]{fontenc}
\usepackage[frenchb]{babel}
\usepackage[top=3cm, bottom=3cm, left=2cm, right=2cm]{geometry}
\usepackage{graphicx}
\usepackage{fancyhdr}
\usepackage{amsmath}

\newcommand{\hmark}{\rule{\linewidth}{0.5mm}}

\begin{document}
\begin{titlepage}
\pagenumbering{gobble}

\centering

\begin{figure}[t]
\begin{center}
\includegraphics[width=8cm]{upmc.png}
\end{center}
\end{figure}

\hmark \\[0.5cm]
\textsc{\textbf{\Large R\'{e}solution de probl\`{e}mes, g\'{e}n\'{e}ration de mots-crois\'{e}s}} \\[0.5cm]
\textsc{Androide M1 -- RP} \\[0.5cm]
\hmark \\[5cm]

\begin{minipage}{0.4\textwidth}
\begin{flushleft}
Thibault \textsc{Gigant} \\
Laura \textsc{Greige}
\end{flushleft}
\end{minipage}
~
\begin{minipage}{0.4\textwidth}
\begin{flushright}
\emph{Enseignants :}\\[0.5cm]
Patrice \textsc{Perny} \\
Morgan \textsc{Chopin}
\end{flushright}
\end{minipage}\\[4cm]

\large 2015 -- 2016

\end{titlepage}
\newpage

\tableofcontents

\newpage

\pagenumbering{arabic}

\pagestyle{fancy}
\renewcommand{\headrulewidth}{1pt}
\lhead{\textbf{RP -- R\'{e}solution de probl\`{e}mes, g\'{e}n\'{e}ration de mots-crois\'{e}s}}
\rhead{}

\section*{Introduction}
\addcontentsline{toc}{section}{Introduction}

\section{Mod\'{e}lisation par un CSP et r\'{e}solution}

Pour r\'{e}soudre ce probl\`{e}me, on peut le mod\'{e}liser comme un probl\`{e}me de satisfaction de contraintes en associant une variable \`{a} chaque mot de la grille. Supposons qu'il y ait $m$ mots dans la grille. \bigskip

\textbf{Variables :} ~ $x_{i}~,~\forall i \in \{1,...,m\}$

\bigskip

\textbf{Domaines :} Soit $dict$ un dictionnaire de mots admissibles:

\begin{align*}
D(x_{i}) = \{ X \in dict \}
\end{align*}

\bigskip

\textbf{Contraintes :} Soit $l_{i}$ la taille du mot en $i$:

\begin{equation} 
len(x_{i}) = l_{i}
\end{equation}
\par
Pour tous mots $x_{i}$ et $x_{j}$ qui se croisent \`{a} la \textit{q}-i\`{e}me lettre de $x_{i}$ et \`{a} la \textit{p}-i\`{e}me lettre de $x_{j}$, on a:
\begin{equation} 
x_{i}[q] = x_{j}[p]
\end{equation}
\par
Si l'on ajoute la contrainte suppl\'{e}mentaire qu'un m\^{e}me mot ne peut appara\^{i}tre plus d'une fois dans la grille, il suffit d'ajouter la contrainte \textit{AllDiff}:
\begin{equation} 
\textit{AllDiff}~(x_{1},x_{2},...x_{m})
\end{equation}

\section{Exp\'{e}rimentation}

\section{Extension au cas pond\'{e}r\'{e}}

\newpage

\section*{Conclusion}
\addcontentsline{toc}{section}{Conclusion}

\end{document}